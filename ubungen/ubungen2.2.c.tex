\documentclass[12pt]{article}
\usepackage[left=1in, right=1in, top=1in, bottom=1in]{geometry}
\usepackage{enumerate}
\usepackage{amsmath, amssymb, amsfonts, amsthm}
\usepackage{physics}
%
\newcommand{\pfrac}[2]{\frac{\partial #1}{\partial #2}}
\newcommand{\drfrac}[2]{\frac{\text{d} #1}{\text{d} #2}}
\def\ep{\varepsilon}
\def\imp{\Rightarrow}
\newcommand{\angstrom}{\text{\normalfont\AA}}
\newcommand{\e}[1]{\times10^{#1}}
%
\setlength{\headheight}{14.5pt}
%
\title{Quantum Computing: A Gentle Introduction}
\author{Brian Benjamin Pomerantz \& Henry Sebastian Graßhorn Gebhardt}
\date{}
%
\begin{document}
\maketitle

\subsection*{2.2}
\begin{itemize}
\item[\textbf{c.}]
Let
\begin{equation*}
\ket{v} = \frac{1}{\sqrt{2}}\left(\ket{0} + \ket{1}\right)
\end{equation*}
and
\begin{equation*}
\ket{v'} = \frac{1}{\sqrt{2}}\left(-\ket{0} + i\ket{1}\right) \,.
\end{equation*}
Measuring with respect to $\ket{v}$,
\begin{equation*}
\abs{\bra{v}\ket{v}}^2 = 1
\end{equation*}
while
\begin{align*}
\bra{v}\ket{v'} &= \left[\frac{1}{\sqrt{2}}\left(\bra{0} + \bra{1}\right)\right]\left[\frac{1}{\sqrt{2}}\left(-\ket{0} + i\ket{1}\right)\right] \\
&= \frac{1}{2}\left(\bra{0} + \bra{1}\right)\left(-\ket{0} + i\ket{1}\right) \\
&= \frac{1}{2}\left(-\bra{0}\ket{0} + i\bra{0}\ket{1} - \bra{1}\ket{0} + \bra{1}\ket{1}\right) \\
&= \frac{1}{2}\left(-1 + 1\right) \\
&= \frac{1}{2}(0) \\
&= 0
\end{align*}
and therefore,
\begin{equation*}
\abs{\bra{v}\ket{v'}}^2 = 0 \neq \abs{\bra{v}\ket{v}}^2 \,.
\end{equation*}

\item[\textbf{h.}]
Note that
\begin{align*}
\ket{v} &= \frac{1}{\sqrt{2}}\left(\ket{i} - \ket{-i}\right) \\
&= \frac{1}{2}\left(\ket{0} + i\ket{1} - \ket{0} + i\ket{1}\right) \\
&= i\ket{1} \,.
\end{align*}
Since $\exists c = e^{i\phi},\ \phi\in\mathbb{R}$ such that $\ket{v} = c\ket{1}$, namely $\phi = \frac{\pi}{2}$, the two states are congruent.
\end{itemize}

\end{document}
